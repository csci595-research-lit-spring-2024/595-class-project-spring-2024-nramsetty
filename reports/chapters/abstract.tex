%Two resources useful for abstract writing.
% Guidance of how to write an abstract/summary provided by Nature: https://cbs.umn.edu/sites/cbs.umn.edu/files/public/downloads/Annotated_Nature_abstract.pdf %https://writingcenter.gmu.edu/guides/writing-an-abstract
\chapter*{\center \Large  Abstract}
%%%%%%%%%%%%%%%%%%%%%%%%%%%%%%%%%%%%%%
% Replace all text with your text
%%%%%%%%%%%%%%%%%%%%%%%%%%%%%%%%%%%

Understanding and predicting human emotions using different computational algorithms and models has become one of the important domains to research as emotion recognition helps in understanding human behaviors. This paper investigates the application of different machine learning algorithms especially deep learning algorithms to understand human emotion based on a photographic image of a human face and predict the emotion where the dataset used to train the model is collected using open web source Kaggle and is an image-based dataset, as the model input will be a different human face. The dataset is annotated with different emotions such as ‘angry’, ‘happy’, ‘sad’, ‘surprise’ etc. There are a total of 7 different emotions that are chosen for the analysis but can be extended if enough image of a particular emotion is collected for training. The primary focus during the entire research is understanding and exploring different Convolutional Neural Networks (CNN) which will be building blocks of the deep learning model as they help in the extraction of different spatial features from the image in a more efficient manner and different evaluation metrics such as accuracy, precision and f1 score will be used to understanding how the model is behaving regarding each emotion as there can be variance bias trade-off. The research also helps in understanding the training and deployment of deep learning models in the real world which can be used in different health care sectors or facial expression devices.


%%%%%%%%%%%%%%%%%%%%%%%%%%%%%%%%%%%%%%%%%%%%%%%%%%%%%%%%%%%%%%%%%%%%%%%%%s

